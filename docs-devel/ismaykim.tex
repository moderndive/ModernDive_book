\documentclass[]{tufte-book}

% ams
\usepackage{amssymb,amsmath}

\usepackage{ifxetex,ifluatex}
\usepackage{fixltx2e} % provides \textsubscript
\ifnum 0\ifxetex 1\fi\ifluatex 1\fi=0 % if pdftex
  \usepackage[T1]{fontenc}
  \usepackage[utf8]{inputenc}
\else % if luatex or xelatex
  \makeatletter
  \@ifpackageloaded{fontspec}{}{\usepackage{fontspec}}
  \makeatother
  \defaultfontfeatures{Ligatures=TeX,Scale=MatchLowercase}
  \makeatletter
  \@ifpackageloaded{soul}{
     \renewcommand\allcapsspacing[1]{{\addfontfeature{LetterSpace=15}#1}}
     \renewcommand\smallcapsspacing[1]{{\addfontfeature{LetterSpace=10}#1}}
   }{}
  \makeatother
\fi

% graphix
\usepackage{graphicx}
\setkeys{Gin}{width=\linewidth,totalheight=\textheight,keepaspectratio}

% booktabs
\usepackage{booktabs}

% url
\usepackage{url}

% hyperref
\usepackage{hyperref}

% units.
\usepackage{units}


\setcounter{secnumdepth}{2}

% citations
\usepackage{natbib}
\bibliographystyle{apalike}
%\renewcommand{\bibsection}{\chapter*{References}}

% pandoc syntax highlighting
\usepackage{color}
\usepackage{fancyvrb}
\newcommand{\VerbBar}{|}
\newcommand{\VERB}{\Verb[commandchars=\\\{\}]}
\DefineVerbatimEnvironment{Highlighting}{Verbatim}{commandchars=\\\{\}}
% Add ',fontsize=\small' for more characters per line
\usepackage{framed}
\definecolor{shadecolor}{RGB}{248,248,248}
\newenvironment{Shaded}{\begin{snugshade}}{\end{snugshade}}
\newcommand{\KeywordTok}[1]{\textcolor[rgb]{0.13,0.29,0.53}{\textbf{{#1}}}}
\newcommand{\DataTypeTok}[1]{\textcolor[rgb]{0.13,0.29,0.53}{{#1}}}
\newcommand{\DecValTok}[1]{\textcolor[rgb]{0.00,0.00,0.81}{{#1}}}
\newcommand{\BaseNTok}[1]{\textcolor[rgb]{0.00,0.00,0.81}{{#1}}}
\newcommand{\FloatTok}[1]{\textcolor[rgb]{0.00,0.00,0.81}{{#1}}}
\newcommand{\ConstantTok}[1]{\textcolor[rgb]{0.00,0.00,0.00}{{#1}}}
\newcommand{\CharTok}[1]{\textcolor[rgb]{0.31,0.60,0.02}{{#1}}}
\newcommand{\SpecialCharTok}[1]{\textcolor[rgb]{0.00,0.00,0.00}{{#1}}}
\newcommand{\StringTok}[1]{\textcolor[rgb]{0.31,0.60,0.02}{{#1}}}
\newcommand{\VerbatimStringTok}[1]{\textcolor[rgb]{0.31,0.60,0.02}{{#1}}}
\newcommand{\SpecialStringTok}[1]{\textcolor[rgb]{0.31,0.60,0.02}{{#1}}}
\newcommand{\ImportTok}[1]{{#1}}
\newcommand{\CommentTok}[1]{\textcolor[rgb]{0.56,0.35,0.01}{\textit{{#1}}}}
\newcommand{\DocumentationTok}[1]{\textcolor[rgb]{0.56,0.35,0.01}{\textbf{\textit{{#1}}}}}
\newcommand{\AnnotationTok}[1]{\textcolor[rgb]{0.56,0.35,0.01}{\textbf{\textit{{#1}}}}}
\newcommand{\CommentVarTok}[1]{\textcolor[rgb]{0.56,0.35,0.01}{\textbf{\textit{{#1}}}}}
\newcommand{\OtherTok}[1]{\textcolor[rgb]{0.56,0.35,0.01}{{#1}}}
\newcommand{\FunctionTok}[1]{\textcolor[rgb]{0.00,0.00,0.00}{{#1}}}
\newcommand{\VariableTok}[1]{\textcolor[rgb]{0.00,0.00,0.00}{{#1}}}
\newcommand{\ControlFlowTok}[1]{\textcolor[rgb]{0.13,0.29,0.53}{\textbf{{#1}}}}
\newcommand{\OperatorTok}[1]{\textcolor[rgb]{0.81,0.36,0.00}{\textbf{{#1}}}}
\newcommand{\BuiltInTok}[1]{{#1}}
\newcommand{\ExtensionTok}[1]{{#1}}
\newcommand{\PreprocessorTok}[1]{\textcolor[rgb]{0.56,0.35,0.01}{\textit{{#1}}}}
\newcommand{\AttributeTok}[1]{\textcolor[rgb]{0.77,0.63,0.00}{{#1}}}
\newcommand{\RegionMarkerTok}[1]{{#1}}
\newcommand{\InformationTok}[1]{\textcolor[rgb]{0.56,0.35,0.01}{\textbf{\textit{{#1}}}}}
\newcommand{\WarningTok}[1]{\textcolor[rgb]{0.56,0.35,0.01}{\textbf{\textit{{#1}}}}}
\newcommand{\AlertTok}[1]{\textcolor[rgb]{0.94,0.16,0.16}{{#1}}}
\newcommand{\ErrorTok}[1]{\textcolor[rgb]{0.64,0.00,0.00}{\textbf{{#1}}}}
\newcommand{\NormalTok}[1]{{#1}}

% longtable
\usepackage{longtable,booktabs}

% multiplecol
\usepackage{multicol}

% strikeout
\usepackage[normalem]{ulem}

% morefloats
\usepackage{morefloats}

% force floats added by CII
\usepackage{float}
\floatplacement{figure}{H}

\let\oldrule=\rule 
\renewcommand{\rule}[1]{\oldrule{\linewidth}}


% tightlist macro required by pandoc >= 1.14
\providecommand{\tightlist}{%
  \setlength{\itemsep}{0pt}\setlength{\parskip}{0pt}}

% title / author / date
\title{ModernDive}

\author{Chester Ismay and Albert Y. Kim}
\date{2017-01-09}

\usepackage{booktabs}
\usepackage{longtable}
\usepackage{framed,color}
\definecolor{shadecolor}{RGB}{248,248,248}
\usepackage{float}

\ifxetex
  \usepackage{letltxmacro}
  \setlength{\XeTeXLinkMargin}{1pt}
  \LetLtxMacro\SavedIncludeGraphics\includegraphics
  \def\includegraphics#1#{% #1 catches optional stuff (star/opt. arg.)
    \IncludeGraphicsAux{#1}%
  }%
  \newcommand*{\IncludeGraphicsAux}[2]{%
    \XeTeXLinkBox{%
      \SavedIncludeGraphics#1{#2}%
    }%
  }%
\fi

%% Need to clean up
\newenvironment{rmdblock}[1]
  {\begin{shaded*}
  \begin{itemize}
  \renewcommand{\labelitemi}{
    \raisebox{-.7\height}[0pt][0pt]{
  %    {\setkeys{Gin}{width=3em,keepaspectratio}\includegraphics{images/#1}}
    }
  }
  \item
  }
  {
  \end{itemize}
  \end{shaded*}
  }
%% Probably can be omitted
\newenvironment{rmdnote}
  {\begin{rmdblock}{note}}
  {\end{rmdblock}}
\newenvironment{rmdcaution}
  {\begin{rmdblock}{caution}}
  {\end{rmdblock}}
\newenvironment{rmdimportant}
  {\begin{rmdblock}{important}}
  {\end{rmdblock}}
\newenvironment{rmdtip}
  {\begin{rmdblock}{tip}}
  {\end{rmdblock}}
\newenvironment{rmdwarning}
  {\begin{rmdblock}{warning}}
  {\end{rmdblock}}
\newenvironment{learncheck}
  {\begin{rmdblock}{warning}}
  {\end{rmdblock}}
\newenvironment{review}
  {\begin{rmdblock}{warning}}
  {\end{rmdblock}}

% To tweak tufte layout
\geometry{
  left=0.8in, % left margin
  textwidth=35pc, % main text block
  marginparsep=1pc, % gutter between main text block and margin notes
  marginparwidth=8pc % width of margin notes
}

\begin{document}

\let\allcaps=\relax
\maketitle



{
\setcounter{tocdepth}{1}
\tableofcontents
}

\chapter{Preamble}\label{preamble}

\section{Principles of this Book}\label{principles-of-this-book}

These are some principles we keep in mind. If you agree with them, this
might be the book for you.

\begin{enumerate}
\def\labelenumi{\arabic{enumi}.}
\tightlist
\item
  \textbf{Blur the lines between lecture and lab}

  \begin{itemize}
  \tightlist
  \item
    Laptops and open source software are rendering the lab/lecture
    dichotomy ever more archaic.
  \item
    It's much harder for students to understand the importance of using
    the software if they only use it once a week or less. They forget
    the syntax in much the same way someone learning a foreign language
    forgets the rules.
  \end{itemize}
\item
  \textbf{Focus on the entire data/science research pipeline}

  \begin{itemize}
  \tightlist
  \item
    Grolemund and Wickham's
    \href{http://r4ds.had.co.nz/introduction.html}{graphic}
  \item
    George Cobb argued for
    \href{https://arxiv.org/abs/1507.05346}{``Minimizing prerequisites
    to research''}
  \end{itemize}
\item
  \textbf{It's all about data, data, data}

  \begin{itemize}
  \tightlist
  \item
    We leverage R packages for rich/complex yet easy-to-load data sets.
  \item
    We've heard it before: ``You can't teach \texttt{ggplot2} for data
    visualization in intro stats!'' We, like
    \href{http://varianceexplained.org/r/teach_ggplot2_to_beginners/}{David
    Robinson}, are more optimistic and we've had success doing so.
  \item
    \texttt{dplyr} is a
    \href{http://chance.amstat.org/2015/04/setting-the-stage/}{game
    changer} for data manipulation: the verb describing your desired
    data action \emph{is} the command name!
  \end{itemize}
\item
  \textbf{Use simulation/resampling for intro stats, not
  probability/large sample approximation}

  \begin{itemize}
  \tightlist
  \item
    Reinforce concepts, not equations, formulas, and probability tables.
  \item
    To this end, we're big fans of the
    \href{https://github.com/ProjectMOSAIC/mosaic}{\texttt{mosaic}}
    package's \texttt{shuffle()}, \texttt{resample()}, and \texttt{do()}
    functions for sampling and simulation.
  \end{itemize}
\item
  \textbf{Don't fence off students from the computation pool, throw them
  in!}

  \begin{itemize}
  \tightlist
  \item
    Don't teach them coding/programming per se, but computation and
    algorithmic thinking.
  \item
    Drawing Venn diagrams delineating statistics, computer science, and
    data science is also ever more archaic; embrace computation!
  \end{itemize}
\item
  \textbf{Complete reproducibility}

  \begin{itemize}
  \tightlist
  \item
    We find it frustrating when textbooks give examples but not the
    source code and the data itself. We not only give you the source
    code for all examples, but also the source code for the whole book!
  \item
    We encourage use of R Markdown to foster notions of reproducible
    research.
  \item
    \textbf{Ultimately the best textbook is one you've written yourself}

    \begin{itemize}
    \tightlist
    \item
      You best know your audience, their background, and their
      priorities and you know best your own style and types of examples
      and problems you like best. Customizability is the ultimate end.
    \item
      A new paradigm for textbooks? Versions, not editions? Pull
      requests, crowd-sourcing, and development versions?
    \end{itemize}
  \end{itemize}
\end{enumerate}

\section{Contribute}\label{contribute}

\begin{itemize}
\tightlist
\item
  This book is in beta testing and is currently at Version 0.1.0.9000.
  If you would like to receive periodic updates on this book and other
  similar projects, please fill out this
  \href{https://goo.gl/forms/IxiwBeEnk72NxMMx2}{Google Form}.
\item
  The source code for this book is available for download/forking on
  \href{https://github.com/ismayc/moderndiver-book}{GitHub}. If you find
  typos or other errors or have suggestions on how to better word
  something in the book, please create a pull request too!
\item
  Please feel free to modify the book as you wish for your own needs!
  All we ask is that you list the authors field above as ``Chester
  Ismay, Albert Y. Kim, and YOU!''
\item
  We'd also appreciate if you let us now what changes you've made and
  how you've used the textbook. We'd love some data on what's working
  well and what's not working so well.
\end{itemize}

\section{Getting Started}\label{getting-started}

This book was written using the \textbf{bookdown} R package from Yihui
Xie. In order to follow along and run the code in this book on your own,
you'll need to have access to R and RStudio. You can find more
information on both of these with a simple Google search for ``R'' and
for ``RStudio.'' An introduction to using R, RStudio, and R Markdown is
also available in a free book
\href{http://ismayc.github.io/rbasics-book}{here} \citep{usedtor2016}.
It is recommended that you refer back to this book frequently as it has
GIF screen recordings that you can follow along with as you learn.

We will keep a running list of R packages you will need to have
installed to complete the analysis as well here in the
\texttt{needed\_pkgs} character vector. You can check if you have all of
the needed packages installed by running all of the lines below. The
last lines including the \texttt{if} will install them as needed (i.e.,
download their needed files from the internet to your hard drive).

You can run the \texttt{library} function on them to load them into your
current analysis. Prior to each analysis where a package is needed, you
will see the corresponding \texttt{library} function in the text. Make
sure to check the top of the chapter to see if a package was loaded
there.

\begin{Shaded}
\begin{Highlighting}[]
\NormalTok{needed_pkgs <-}\StringTok{ }\KeywordTok{c}\NormalTok{(}\StringTok{"nycflights13"}\NormalTok{, }\StringTok{"dplyr"}\NormalTok{, }\StringTok{"ggplot2"}\NormalTok{, }\StringTok{"knitr"}\NormalTok{, }
  \StringTok{"okcupiddata"}\NormalTok{, }\StringTok{"dygraphs"}\NormalTok{, }\StringTok{"rmarkdown"}\NormalTok{, }\StringTok{"mosaic"}\NormalTok{, }\StringTok{"ggplot2movies"}\NormalTok{)}

\NormalTok{new.pkgs <-}\StringTok{ }\NormalTok{needed_pkgs[!(needed_pkgs %in%}\StringTok{ }\KeywordTok{installed.packages}\NormalTok{())]}

\NormalTok{if(}\KeywordTok{length}\NormalTok{(new.pkgs)) \{}
  \KeywordTok{install.packages}\NormalTok{(new.pkgs, }\DataTypeTok{repos =} \StringTok{"http://cran.rstudio.com"}\NormalTok{)}
\NormalTok{\}}
\end{Highlighting}
\end{Shaded}

\section*{Colophon}\label{colophon}
\addcontentsline{toc}{section}{Colophon}

The source of the book is available
\href{https://github.com/ismayc/moderndiver-book}{here} and was built
with versions of R packages (and their dependent packages) given below.
This may not be of importance for initial readers of this book, but the
hope is you can reproduce a duplicate of this book by installing these
versions of the packages.

\begin{longtable}{lllll}
\toprule
package & * & version & date & source\\
\midrule
assertthat &  & 0.1 & 2013-12-06 & CRAN (R 3.3.0)\\
backports &  & 1.0.4 & 2016-10-24 & cran (@1.0.4)\\
base64enc &  & 0.1-3 & 2015-07-28 & CRAN (R 3.3.0)\\
BH &  & 1.62.0-1 & 2016-11-19 & CRAN (R 3.3.2)\\
bitops &  & 1.0-6 & 2013-08-17 & CRAN (R 3.3.0)\\
\addlinespace
caTools &  & 1.17.1 & 2014-09-10 & CRAN (R 3.3.0)\\
colorspace &  & 1.3-2 & 2016-12-14 & CRAN (R 3.3.2)\\
curl &  & 2.3 & 2016-11-24 & CRAN (R 3.3.2)\\
DBI &  & 0.5-1 & 2016-09-10 & CRAN (R 3.3.0)\\
dichromat &  & 2.0-0 & 2013-01-24 & CRAN (R 3.3.0)\\
\addlinespace
digest &  & 0.6.11 & 2017-01-03 & CRAN (R 3.3.2)\\
dplyr &  & 0.5.0 & 2016-06-24 & CRAN (R 3.3.0)\\
dygraphs &  & 1.1.1.4 & 2017-01-04 & CRAN (R 3.3.2)\\
evaluate &  & 0.10 & 2016-10-11 & CRAN (R 3.3.0)\\
ggdendro &  & 0.1-20 & 2016-04-27 & cran (@0.1-20)\\
\addlinespace
ggplot2 &  & 2.2.1 & 2016-12-30 & CRAN (R 3.3.2)\\
ggplot2movies &  & 0.0.1 & 2015-08-25 & CRAN (R 3.3.0)\\
gridExtra &  & 2.2.1 & 2016-02-29 & CRAN (R 3.3.0)\\
gtable &  & 0.2.0 & 2016-02-26 & CRAN (R 3.3.0)\\
highr &  & 0.6 & 2016-05-09 & CRAN (R 3.3.0)\\
\addlinespace
hms &  & 0.3 & 2016-11-22 & CRAN (R 3.3.2)\\
htmltools &  & 0.3.5 & 2016-03-21 & CRAN (R 3.3.0)\\
htmlwidgets &  & 0.8 & 2016-11-09 & CRAN (R 3.3.2)\\
jsonlite &  & 1.2 & 2016-12-31 & CRAN (R 3.3.2)\\
knitr &  & 1.15.1 & 2016-11-22 & CRAN (R 3.3.2)\\
\addlinespace
labeling &  & 0.3 & 2014-08-23 & CRAN (R 3.3.0)\\
lattice &  & 0.20-34 & 2016-09-06 & CRAN (R 3.3.2)\\
latticeExtra &  & 0.6-28 & 2016-02-09 & cran (@0.6-28)\\
lazyeval &  & 0.2.0 & 2016-06-12 & CRAN (R 3.3.0)\\
magrittr &  & 1.5 & 2014-11-22 & CRAN (R 3.3.0)\\
\addlinespace
markdown &  & 0.7.7 & 2015-04-22 & CRAN (R 3.3.0)\\
MASS &  & 7.3-45 & 2016-04-21 & CRAN (R 3.3.2)\\
Matrix &  & 1.2-7.1 & 2016-09-01 & CRAN (R 3.3.2)\\
mime &  & 0.5 & 2016-07-07 & CRAN (R 3.3.0)\\
mosaic &  & 0.14.4 & 2016-11-05 & Github (ProjectMOSAIC/mosaic@c0b1f10)\\
\addlinespace
mosaicData &  & 0.14.0 & 2016-06-17 & cran (@0.14.0)\\
munsell &  & 0.4.3 & 2016-02-13 & CRAN (R 3.3.0)\\
nycflights13 &  & 0.2.1 & 2016-12-30 & CRAN (R 3.3.2)\\
okcupiddata &  & 0.1.0 & 2016-08-19 & local\\
plyr &  & 1.8.4 & 2016-06-08 & CRAN (R 3.3.0)\\
\addlinespace
R6 &  & 2.2.0 & 2016-10-05 & CRAN (R 3.3.0)\\
RColorBrewer &  & 1.1-2 & 2014-12-07 & CRAN (R 3.3.0)\\
Rcpp &  & 0.12.8 & 2016-11-17 & CRAN (R 3.3.2)\\
readr &  & 1.0.0 & 2016-08-03 & CRAN (R 3.3.0)\\
reshape2 &  & 1.4.2 & 2016-10-22 & CRAN (R 3.3.0)\\
\addlinespace
rmarkdown &  & 1.3 & 2016-12-21 & CRAN (R 3.3.2)\\
rprojroot &  & 1.1 & 2016-10-29 & cran (@1.1)\\
scales &  & 0.4.1 & 2016-11-09 & CRAN (R 3.3.2)\\
stringi &  & 1.1.2 & 2016-10-01 & CRAN (R 3.3.0)\\
stringr &  & 1.1.0 & 2016-08-19 & CRAN (R 3.3.0)\\
\addlinespace
tibble &  & 1.2 & 2016-08-26 & CRAN (R 3.3.0)\\
tidyr &  & 0.6.0 & 2016-08-12 & CRAN (R 3.3.0)\\
xts &  & 0.9-7 & 2014-01-02 & CRAN (R 3.3.0)\\
yaml &  & 2.1.14 & 2016-11-12 & CRAN (R 3.3.2)\\
zoo &  & 1.7-14 & 2016-12-16 & CRAN (R 3.3.2)\\
\bottomrule
\end{longtable}

\textbf{Book was last updated:}

\begin{verbatim}
## [1] "By aykim on Monday, January 09, 2017 00:30:34 PST"
\end{verbatim}

\chapter{Introduction}\label{intro}

\part{Data Exploration}\label{part-data-exploration}

\chapter{\texorpdfstring{Data Visualization via
\texttt{ggplot2}}{Data Visualization via ggplot2}}\label{data-visualization-via-ggplot2}

\chapter{\texorpdfstring{Data Manipulation via
\texttt{dplyr}}{Data Manipulation via dplyr}}\label{data-manipulation-via-dplyr}

\part{Inference}\label{part-inference}

\chapter{Hypothesis Testing}\label{hypo}

\chapter{Confidence Intervals}\label{ci}

\chapter{\texorpdfstring{Regression via
\texttt{broom}}{Regression via broom}}\label{regression-via-broom}

\part{Conclusion}\label{part-conclusion}

\appendix


\chapter{Inference Examples}\label{appendixB}

\chapter{Reach for the Starts}\label{appendixC}

\chapter{Placeholder}\label{placeholder}

\renewcommand{\bibname}{References}
\addcontentsline{toc}{chapter}{References}
\bibliography{bib/packages.bib,bib/books.bib,bib/articles.bib}
% 
% 

\end{document}
